
\documentclass[letterpaper,reqno,12pt]{article}
\usepackage[margin=1.0in]{geometry}
\usepackage{color,latexsym,amsmath,amssymb}
\usepackage{fancyhdr}
\usepackage{amsthm}
\usepackage[linesnumbered,lined,boxed,commentsnumbered,noend,noline]{algorithm2e}
\usepackage{dsfont}
\usepackage{graphicx}
\usepackage{hyperref}
\usepackage{lmodern}
\usepackage[numbers]{natbib}
\usepackage{listings}% http://ctan.org/pkg/listings
\lstset{
  basicstyle=\ttfamily,
  columns=fullflexible,
  mathescape
}

\setlength{\AlCapSkip}{0.5em}

\allowdisplaybreaks

\newcommand{\RR}{\mathbb{R}}
\newcommand{\CC}{\mathbb{C}}
\newcommand{\ZZ}{\mathbb{Z}}
\newcommand{\QQ}{\mathbb{Q}}
\newcommand{\NN}{\mathbb{N}}
\newcommand{\mynote}[3][red]
  {{\color{#1} \fbox{\bfseries\sffamily\scriptsize#2}
  {\small$\blacktriangleright$\textsf{\emph{#3}}$\blacktriangleleft$}}~}
\newcommand{\yp}[1]{\mynote{YP}{#1}}
\DeclareMathOperator{\conv}{conv}
\DeclareMathOperator{\charcone}{char.cone}
\DeclareMathOperator{\STAB}{STAB}
\DeclareMathOperator{\Down}{Down}
\DeclareMathOperator{\lca}{lca}
\DeclareMathOperator{\LPO}{LPO}
\DeclareMathOperator{\OPT}{OPT}
\DeclareMathOperator{\LHS}{LHS}
\DeclareMathOperator{\RHS}{RHS}
\DeclareMathOperator{\tr}{tr}
\DeclareMathOperator{\vol}{vol}
\DeclareMathOperator{\argmin}{arg\,min}
\DeclareMathOperator{\argmax}{arg\,max}
\DeclareMathOperator{\poly}{poly}
\begin{document}
\pagenumbering{arabic}
\title{Network Flow Algorithms: Exercise Solutions}
\author{Yuchong Pan}
\date{\today}
\newtheorem{theorem}{Theorem}
\newtheorem{lemma}{Lemma}
\newtheorem{claim}{Claim}
\theoremstyle{definition} \newtheorem{exercise}{Exercise} [section]
\theoremstyle{definition} \newtheorem{defn}{Definition}
\maketitle
%

\section{Preliminaries: Shortest Path Algorithms}

\begin{exercise}
  Let $i_k$ be the vertex selected at the $k^\text{th}$ iteration of Dijkstra's algorithm. We prove by induction that at the beginning of the $k^\text{th}$ iteration, $d(v) \in \{ d(i_k), d(i_k) + 1, \infty \}$ for all $v \in V$ not marked yet. At the beginning of the first iteration, $s$ is selected, and any $v \in V$ has $d(v) = \infty$; this proves the base case.
  
  Suppose that at the beginning of the $k^\text{th}$ iteration, $d(v) \in \{ d(i_k), d(i_k) + 1, \infty \}$ for all $v \in V$ not marked yet. Let $(i_k, j) \in A$ such that $j$ is not marked yet. If $d(i_k) + c(i_k, j) = d(i_k) + 1 < d(j)$, then we set $d(j) \leftarrow d(i_k) + 1$; otherwise, $d(j)$ remains the same, and hence $d(j) \in \{ d(i_k), d(i_k) + 1, \infty \}$. If $d(i_{k + 1}) = d(i_k)$, then we are done; otherwise, $d(i_{k + 1}) = d(i_k) + 1$, and $d(v) \in \{ d(i_k) + 1, \infty \} = \{ d(i_{k + 1}), \infty \}$ for all $v \in V$ not marked yet. This completes the induction step.

  Now, consider the $k^\text{th}$ iteration. If $d(i_k) + c(i_k, j) = d(i_k) + 1 < d(j)$ for some $(i, j) \in A$, then $d(j)$ sas $\infty$, and we set $d(j) \leftarrow d(i_k) + 1$. Since $d(v) \in \{ d(i_k), d(i_k) + 1, \infty \}$ for all $v \in V$ not marked yet, then we can process $j$ after any $v \in V$ not marked yet such that $d(v) < \infty$ is processed. Therefore, we can maintain a queue of all $v \in V$ not marked yet such that $d(v) < \infty$, and we push $j$ to the tail of the queue if $d(j)$ is updated. The adapted algorithm is given in Algorithm \ref{alg:adapted-dijkstra}. It is clear that Algorithm \ref{alg:adapted-dijkstra} runs in $O(m)$ time.

  \begin{algorithm}
    $q \leftarrow \textit{new queue()}$ \\
    $d(i) \leftarrow \infty$ for all $i \in V$ \\
    $p(i) \leftarrow \textbf{null}$ for all $i \in V$ \\
    $d(s) \leftarrow 0$ \\
    $q.\textit{push($s$)}$ \\
    \While{not $q.\textit{empty?}$}{
      $i \leftarrow q.\textit{front()}$ \\
      \For{$j \in V$ such that $(i, j) \in A$}{
        \If{$d(j) > d(i) + 1$}{
          $d(j) \leftarrow d(i) + 1$ \\
          $p(j) \leftarrow i$ \\
          $q.\textit{push(j)}$
        }
      }
    }
    \caption{Adapted Dijkstra's algorithm where $c(i, j) = 1$ for all $(i, j) \in A$.}
    \label{alg:adapted-dijkstra}
  \end{algorithm}
\end{exercise}

\end{document}
